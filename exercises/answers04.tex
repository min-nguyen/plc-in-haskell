
%       dvips -ta4 exercise04 -o
%       ps2pdf -sPAPERSIZE=a4 exercise04.ps

\documentclass[a4paper]{article}
\usepackage{times,fullpage}
\usepackage[latin1]{inputenc}
\usepackage[T1]{fontenc}
\usepackage{bprd}
\usepackage[dvips]{graphicx}

\setcounter{exercise-sheet}{4}

\begin{document}

\begin{center}
{\Large\bf Exercises week \arabic{exercise-sheet}\\[1ex]
Monday 14 September 2009}\\[1ex]
% {\small 2009-09-10}\\[2ex]
\end{center}

\subsection*{Goal of the exercises}

The goal of this week's exercises is to understand the evaluation of a
simple first-order functional language, and how explicit types can be
given and checked.

Do exercises~\ref{exer-fun-build}, \ref{exer-fun-more-examples},
\ref{exer-fun-multiple-arguments},
\ref{exer-fun-multiple-arguments-parse},
\ref{exer-fun-sequential-logical} and
\ref{exer-fun-multiple-arguments-typed}, and hand in the latter four.

\begin{exercise}\label{exer-fun-build}
  Get archive \texttt{fun.zip} from the homepage and unpack to
  directory \texttt{FunLang}\@.  It contains lexer and parser
  specifications and interpreter for a small first-order functional
  language.  Generate and compile the lexer and parser by running
  \texttt{alex FunLex.x} and \texttt{happy FunPar.y}\@.
  
  Parse and run some example programs using
  file \texttt{Fun/ParseAndRun.hs} by running \texttt{ghci ParseAndRun.hs Fun.hs Absyn.hs Parse.hs FunLex.hs FunPar.hs} and then calling \texttt{runProgram} on an example program as a string.
  
  
  Alternatively, run \texttt{ghci Parse.hs Absyn.hs FunLex.hs FunPar.hs}, and then parse some premade programs found in \texttt{Fun/Parse.hs}; you can do this by simply entering \texttt{e1} up to \texttt{e5}.
\end{exercise}


\begin{exercise}\label{exer-fun-more-examples}
  Write more example programs in the functional language.  Then test
  them in the same way as in Exercise~\ref{exer-fun-build}\@.  For
  instance, write programs that do the following:

  \begin{itemize}
  \item Compute the sum of the numbers from 1000 down to 1.  Do this
    by defining a function \texttt{sum n} that computes the sum
    $n+(n-1)+\cdots + 2 + 1$.  (Use straightforward summation, no
    clever tricks).

  \item Compute the number $3^8$, that is, 3 raised to the power 8.
    Again, use a recursive function.

  \item Compute $3^0 + 3^1 + \cdots + 3^{10} + 3^{11}$, using two
    recursive functions.

  \item Compute $1^8 + 2^8 + \cdots + 10^8$, again using two recursive
    functions.
  \end{itemize}
\end{exercise}


\begin{exercise}\label{exer-fun-multiple-arguments}
  For simplicity, the current implementation of the functional
  language requires all functions to take exactly one argument.  This
  seriously limits the programs that can be written in the language
  (at least it limits what that can be written without excessive
  cleverness and complications).
  
  Modify the language to permit functions to take one or more
  arguments.  Start by modifying the abstract syntax in
  \texttt{FunLang/Absyn.hs} to permit a list of parameter names in
  \texttt{Letfun} and a list of argument expressions in
  \texttt{Call}\@.  \\
  
  
\noindent
\textbf{\emph{Answer}} 
{\codesetup\begin{verbatim}
data Expr = ...
          | Letfun String [String] Expr Expr
          | Call Expr [Expr]
\end{verbatim}} 

  
  Then modify the \texttt{eval} interpreter in file
  \texttt{FunLang/Fun.hs} to work for the new abstract syntax.  You must
  modify the closure representation to accommodate a list of
  parameters.  Also, modify the \texttt{Letfun} and \texttt{Call}
  clauses of the interpreter.  You will need a way to combine together a
  list of variable names and a list of variable values, to get an
  environment in the form of an association list; so function
  \texttt{zip} might be useful.\\
  
  
\noindent
\textbf{\emph{Answer}} 
{\codesetup\begin{verbatim}
data Value = ...
           | Closure String [String] Expr (Env Value)
           
eval :: Expr -> Env Value -> Int
eval (Letfun f xs fBody letBody) env =
    let bodyEnv = ((f, Closure f xs fBody env) : env)
    in eval letBody bodyEnv
eval (Call (Var f) eArgs) env = 
    let fClosure = lookup env f
    in case fClosure of 
        Closure f xs fBody fDeclEnv ->
           let xVals = map (Num . flip eval env) eArgs
               varArgs = zip xs xVals
               fBodyEnv = ((f, fClosure): (varArgs ++ fDeclEnv))
           in eval fBody fBodyEnv
\end{verbatim}} 

  
\end{exercise}

\begin{exercise}\label{exer-fun-multiple-arguments-parse}
  In continuation of Exercise~\ref{exer-fun-multiple-arguments},
  modify the parser specification to accept a language where functions
  may take any (non-zero) number of arguments.  The resulting parser
  should permit function declarations such as these:

{\codesetup\begin{verbatim}
  let pow x n = if n=0 then 1 else x * pow x (n-1) in pow 3 8 end

  let max2 a b = if a<b then b else a
  in let max3 a b c = max2 a (max2 b c)
     in max3 25 6 62 end
  end
\end{verbatim}}


\noindent 
You may want to define non-empty parameter lists and argument lists.  Note that multi-argument applications such as\ \ 
\texttt{f a b}\ \ are already permitted by the existing grammar, but
they would produce abstract syntax of the form \texttt{Call(Call(Var
  "f", Var "a"), Var "b")} which the \texttt{Fun.eval} function does
not understand.  You need to modify the AppExpr nonterminal and its
semantic action to produce \texttt{Call(Var "f", [Var "a"; Var "b"])}
instead.\\

\noindent
\textbf{\emph{Answer (FunPar.y)}} 
{\codesetup\begin{verbatim}
AtExpr  : ...
        | let name Args '=' Expr in Expr end    { Letfun $2 $3 $5 $7    }
        | '(' Expr ')'                          { $2                    }

Args    : name                                  { $1    }
        | name Args                             { $1 $2 }
        
AppExpr : AppExpr AppArgs                       { Call $1 $2            }
  
AppArgs : AtExpr AppArgs                        {  $1 $2                }
        | AtExpr                                {  $1                   }
\end{verbatim}}

\end{exercise}


\begin{exercise}\label{exer-fun-sequential-logical}
  Extend the (untyped) functional language with infix operator
  `\verb|&&|' meaning sequential logical `and' and infix operator
  `\verb+||+' meaning sequential logical `or', as in C, C++, Java,
  C\#, F\#\@.  Note that \verb+e1 && e2+ can be encoded as \texttt{if
    e1 then e2 else false} and that \verb+e1 || e2+ can be encoded as
  \texttt{if e1 then true else e2}.  Hence you need only change the
  lexer and parser specifications, and make the new rules in the
  parser specification generate the appropriate abstract syntax.  You
  need not change \texttt{FunLang/Absyn.hs} or \texttt{FunLang/Fun.hs}\@.\\
  
  
\noindent
\textbf{\emph{Answer (FunLex.x)}} 
{\codesetup\begin{verbatim}

tokens :-

    ...
    (\|\||\&\&)          { \s -> let (x:y:ys) = s in operator [x,y])  }

operator :: String -> Token
operator c = case c of ... 
                      "&&" -> TokenAnd
                      "||" -> TokenOr
                      
data Token = ...
           | TokenAnd
           | TokenOr
           | ...
\end{verbatim}}

\noindent
\textbf{\emph{Answer (FunPar.y)}} 
{\codesetup\begin{verbatim}

%token  
        ...    
        "&&"    { TokenAnd              }
        "||"    { TokenOr               }



Expr    : ....
        | Expr '&&' Expr                         { Prim "&&" $1 $3        }
        | Expr '||' Expr                         { Prim "||" $1 $3        }
        ....
\end{verbatim}}

\end{exercise}

\begin{exercise}\label{exer-fun-tuples-eval}
  Extend the abstract syntax, the concrete syntax, and the interpreter
  for the untyped functional language to handle tuple constructors
  \texttt{(...)} and component selectors \verb+#i+ such that \verb+#i+ selects the ith component in a tuple.

{\codesetup\begin{verbatim}
   data Expr = 
     | ...
     | Tup [Expr] 
     | Sel Int Expr
     | ...
\end{verbatim}}

\noindent
If we use the concrete syntax \verb+#2(e)+ for \texttt{Sel 2 e} and
\texttt{(e1, e2)} for \texttt{Tup [e1, e2]} then you should be able to
write programs such as these:

{\codesetup\begin{verbatim}
   let t = (1+2, false, 5>8)
   in if #3(t) then #1(t) else 14 end
\end{verbatim}}

\noindent
and

{\codesetup\begin{verbatim}
   let max xy = if #1(xy) > #2(xy) then #1(xy) else #2(xy) 
   in max (3, 88) end
\end{verbatim}}

\noindent
This permits functions to take multiple arguments and return multiple
results.

To extend the interpreter correspondingly, you need to introduce a data type for values. This also should take care of tuple values \texttt{TupV vs}. This allows the function
\texttt{eval} to return results of different value types, (not just integers!):


{\codesetup\begin{verbatim}
   data Value = 
     -- integer value
       IntV Int     
     -- tuple value
     | TupV [Value]
     -- (funct name, args, funct body, func environment)
     | Closure String [String] Expr (Env Value) 
   
   eval :: Expr -> Env Value -> Value
   eval = ...
\end{verbatim}}

\noindent 
Note that this requires some changes elsewhere in the \texttt{eval}
interpreter.  For instance, the primitive operations currently work
because \texttt{eval} always returns an \texttt{Int}, but with the
suggested change, they will have to check that \texttt{eval} returns
an \texttt{IntV i} (e.g.\ by pattern matching).\\

\noindent
\textbf{\emph{Answer}} 
{\codesetup\begin{verbatim}
eval :: Expr -> Env Value -> Value
eval (CstI i) env = IntV i 
eval (CstB b) env = IntV (if b then 1 else 0) 
eval (Var  x) env = 
    case lookup env x of 
      IntV i -> IntV i 
      _     -> error "eval Var"
eval (Prim ope e1 e2) env = 
    let IntV i1 = eval e1 env
        IntV i2 = eval e2 env
    in case ope of
        "*" -> IntV $ i1 * i2
        "+" -> IntV $ i1 + i2
        "-" -> IntV $ i1 - i2
        "=" -> IntV $ if i1 == i2 then 1 else 0
        "<" -> IntV $ if i1 < i2 then 1 else 0
        _   -> error ("unknown primitive " ++ ope)
eval (Let x eRhs letBody) env = 
    let xVal    = Num (eval eRhs env)
        bodyEnv = ((x, xVal):env)
    in  eval letBody bodyEnv
eval (If e1 e2 e3) env = 
    let IntV b = eval e1 env
    in  if b /= 0 
        then eval e2 env
        else eval e3 env
eval (Letfun f x fBody letBody) env =
    let bodyEnv = ((f, Closure f x fBody env) : env)
    in  eval letBody bodyEnv
eval (Call (Var f) eArg) env = 
    let fClosure = lookup env f
    in  case fClosure of
         Closure f x fBody fDeclEnv ->
            let xVal = IntV (eval eArg env)
                fBodyEnv = ((x, xVal):(f, fClosure):fDeclEnv)
            in  eval fBody fBodyEnv
         _ -> error "eval Call: not a function"
eval (Tup es) env = TupV (map (flip eval env) es)
eval (Sel i e) env = eval (e !! i) env
eval (Call _ _) env = error "eval Call: not first-order function"

\end{verbatim}}
\end{exercise}


\begin{exercise}\label{exer-fun-multiple-arguments-typed} 
  Modify the abstract syntax \texttt{TyExpr} and the type checker
  functions \texttt{typ} and \texttt{typeCheck} in
  \texttt{TypedFun/TypedFun.hs} to allow functions to take any number
  of typed parameters.  
  
  This exercise is similar to
  Exercise~\ref{exer-fun-multiple-arguments}, but concerns the typed
  language.  The changes to the interpreter function \texttt{eval} are
  very similar to those in Exercise~\ref{exer-fun-multiple-arguments}
  and can be omitted; just delete the \texttt{eval} function.\\
  
  
\noindent
\textbf{\emph{Answer}} 
{\codesetup\begin{verbatim}
data TyExpr = ...
            | Letfun String [(String, Typ)] TyExpr Typ TyExpr
\end{verbatim}}
\end{exercise}


\begin{exercise}
  Add lists (\texttt{CstN} is the empty list \texttt{[]},
  \texttt{ConC e1 e2} is \texttt{(e1:e2)}), and list pattern matching
  expressions to the typed functional language, where
  \texttt{Match e0 e1 (h,t, e2))} is \texttt{match e0 with [] -> e1
    | h:t -> e2}\\

\noindent
\textbf{\emph{Answer}} 
{\codesetup\begin{verbatim}
   data TyExpr = 
     | ...
     | CstN
     | ConC of TyExpr TyExpr
     | Match TyExpr TyExpr (String, String, TyExpr)
     | ..
\end{verbatim}}

\end{exercise}

\begin{exercise}
  Add type checking for lists.  All elements of a list must have the
  same type.  You'll need a new kind of type constructor \texttt{TypL Typ} to
  represent the type of lists with elements of a given type. In the case of type checking an empty list, for simplicity, also introduce a type constructor \texttt{TypPoly} to indicate (uninstantiated) polymorphic types.\\
  
\noindent
\textbf{\emph{Answer}} 
{\codesetup\begin{verbatim}
data Typ = TypL Typ
         | TypNil
         | ...
         
typ :: TypExpr -> Env Typ -> Typ
typ (CstN) env = TypL TypPoly
typ (ConC x CstN) env = 
    let t1 = typ x env
    in TypL t1
typ (ConC x xs) env = 
    let t1 = typ x env
        t2 = typ xs env
    in if TypL t1 == t2
    then TypL t1
    else error "badly typed list"
\end{verbatim}}
\end{exercise}

\end{document}
%       dvips -ta4 exercise02 -o
%       ps2pdf -sPAPERSIZE=a4 exercise02.ps

\documentclass[a4paper]{article}
\usepackage{times,fullpage}
\usepackage[latin1]{inputenc}
\usepackage[T1]{fontenc}
\usepackage{bprd}
\usepackage[dvips]{graphicx}
\usepackage{xcolor}

\setcounter{exercise-sheet}{16}

\begin{document}

\begin{center}
{\Large\bf Exercises week \arabic{exercise-sheet}}\\[1ex]
% {\small 2009-08-28}\\[2ex]
\end{center}


\begin{exercise}\label{exer-right-linear}
  Define an Haskell function \texttt{linear ::\ Int -> Tree Int} so that
  \texttt{linear $n$} produces a right-linear tree with $n$ nodes.
  For instance, \texttt{linear 0} should produce \texttt{Lf}, and
  \texttt{linear 2} should produce \texttt{Br 2 Lf (Br 1 Lf Lf)}.  The definition
  of \texttt{Tree} is given below.
{\codesetup\begin{verbatim}
   data Tree a = Br a (Tree a) (Tree a) | Lf
\end{verbatim}}
\noindent
\textbf{\emph{Answer}}
{\codesetup\begin{verbatim}
   linear :: Int -> Tree Int
   linear n = if n == 0
              then Lf 
              else Br n Lf (linear (n-1))
\end{verbatim}}

\end{exercise}


\begin{exercise}\label{exer-tree-traversal}
  Lecture 2 presents an Haskell function \texttt{preorder1 ::\ Tree a ->
    [a]} that returns a list of the node values in a tree, in
  \emph{preorder} (root before left subtree before right subtree).
  
  Now define a function \texttt{inorder} that returns the node values
  in \emph{inorder} (left subtree before root before right subtree)
  and a function \texttt{postorder} that returns the node values in
  \emph{postorder} (left subtree before right subtree before root):\\
  
\noindent
\textbf{\emph{Answer}}
{\codesetup\begin{verbatim}
   inorder   : Tree a -> [a]
   inorder Lf = []
   inorder (Br x l r) = inorder l ++ [x] ++ inorder r
   
   postorder : Tree a -> [a]
   postorder Lf = []
   postorder (Br x l r) = postorder l ++ postorder r ++ [x]
\end{verbatim}}

\noindent 
Finally, define a more efficient version of \texttt{inorder} that
  uses an auxiliary function \texttt{ino ::\ Tree a -> [a] -> [a]
    } with an accumulating parameter; and similarly for
  \texttt{postorder}.  \\
  
  \noindent
\textbf{\emph{Answer}}
{\codesetup\begin{verbatim}
   inorder   : Tree a -> [a]
   inorder Lf = []
   inorder (Br x l r) = inorder l ++ [x] ++ inorder r
   
   postorder : Tree a -> [a]
   postorder Lf = []
   postorder (Br x l r) = postorder l ++ postorder r ++ [x]
\end{verbatim}}
\end{exercise}


\begin{exercise}\label{exer-multi-let-eval}
  Extend the expression language \texttt{Expr} from \texttt{Intcomp1.hs}
  with multiple \emph{sequential} let-bindings, such as this (in
  concrete syntax):

{\codesetup\begin{verbatim}
   let x1 = 5+7   x2 = x1*2 in x1+x2 end
\end{verbatim}}

\noindent
\textbf{\emph{Answer}}\\
An example solution for \texttt{Expr} is: 

{\codesetup\begin{verbatim}
   data Expr = 
     | CstI Int
     | Var String
     | Let [(String, Expr)] Expr    
     | Prim String Expr Expr
     deriving Show
\end{verbatim}}

\noindent 
so that the \texttt{Let} constructor takes a list of bindings, where a
binding is a pair of a variable name and an expression.  The example
above would be represented as:

{\codesetup\begin{verbatim}
   Let [("x1", ...), ("x2", ...)] (Prim "+" (Var "x1") (Var "x2"))
\end{verbatim}}

\vspace{0.3cm}
\noindent 
Revise the \texttt{eval} interpreter from \texttt{Intcomp1.hs} to work
for the \texttt{Expr} language extended with multiple sequential
let-bindings.  \\

\noindent
\textbf{\emph{Answer}}
{\codesetup\begin{verbatim}
   eval :: Expr -> [(String, Int)] -> Int 
   eval (Let xs ebody) env
       = let env1 = foldl f env xs
         in  eval ebody env1
         where f env' (x, xexpr) = let xval = eval xexpr env'
                                   in  ((x, xval):env')
\end{verbatim}}

\end{exercise}


\begin{exercise}\label{exer-multi-let-closed}
  Revise the function \texttt{freevars ::\ Expr -> [String]} to work
  for the language as extended in Exercise~\ref{exer-multi-let-eval}.
  Note that the example expression in the beginning of
  Exercise~\ref{exer-multi-let-eval} has no free variables, but
  \texttt{let x1 = x1+7 in x1+8 end} has the free variable
  \texttt{x1}, because the variable \texttt{x1} is bound only in the
  body (\texttt{x1+8}), not in the right-hand side (\texttt{x1+7}), of
  its own binding.  (There \emph{are} programming languages where a
  variable can be used in the right-hand side of its own binding, but
  this is not such a language.)\\
  
\noindent
\textbf{\emph{Answer}}
{\codesetup\begin{verbatim}
  freevars :: Expr -> [String]
  freevars (CstI i) = []
  freevars (Var x)  = [x]
  freevars (Let xs ebody) 
      = foldr f (freevars ebody) xs 
      where f (x, erhs) freevars'  = union (freevars erhs, minus (freevars', [x]))
  freevars (Prim op e1 e2) 
      = union (freevars e1, freevars e2)
\end{verbatim}}
\end{exercise}








\begin{exercise}\label{exer-simultaneous-let}
  
  Now modify the interpretation of the language from
  Exercise~\ref{exer-multi-let-eval} so that multiple let-bindings are
  \emph{simultaneous} rather than sequential.  For instance,

{\codesetup\begin{verbatim}
   let x1 = 5+7   x2 = x1*2 in x1+x2 end
\end{verbatim}}

\noindent
should still have the abstract syntax 

{\codesetup\begin{verbatim}
   Let [("x1", ...); ("x2", ...)] (Prim "+" (Var "x1") (Var "x2"))
\end{verbatim}}

\noindent 
but now the interpretation is that all right-hand sides must be
evaluated before any left-hand side variable gets bound to its
right-hand side value.  That is, in the above expression, the
occurrence of \texttt{x1} in the right-hand side of \texttt{x2} has
nothing to do with the \texttt{x1} of the first binding; it is a free
variable.

Revise the \texttt{eval} interpreter to work for this version of the
\texttt{Expr} language.  The idea is that all the right-hand side
expressions should be evaluated, after which all the variables are
bound to those values simultaneously.  Hence

{\codesetup\begin{verbatim}
   let x = 11 in let x = 22  y = x+1 in x+y end end 
\end{verbatim}}

\noindent 
should compute \texttt{12 + 22} because \texttt{x} in \texttt{x+1} is
the outer \texttt{x} (and hence is 11), and \texttt{x} in \texttt{x+y}
is the inner \texttt{x} (and hence is 22).  In other words, in the
let-binding

{\codesetup\begin{verbatim}
   let x1 = e1  ...  xn = en in e end
\end{verbatim}}

\noindent 
the scope of the variables \texttt{x1 \ldots\ xn} should be
\texttt{e}, not \texttt{e1 \ldots\ en}.  
\end{exercise}
\noindent
\textbf{\emph{Answer}}
{\codesetup\begin{verbatim}
   eval :: Expr -> [(String, Int)] -> Int 
   eval (Let xs ebody) env
       = let (env1, _) = foldl f (env, env) xs
         in  eval ebody env1
         where f (env', env) (x, xexpr) = let xval = eval xexpr env
                                          in  ((x, xval):env'), env)
\end{verbatim}}


\end{document}


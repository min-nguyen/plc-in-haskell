%       dvips -ta4 exercise02 -o
%       ps2pdf -sPAPERSIZE=a4 exercise02.ps

\documentclass[a4paper]{article}
\usepackage{times,fullpage}
\usepackage[latin1]{inputenc}
\usepackage[T1]{fontenc}
\usepackage{bprd}
\usepackage[dvips]{graphicx}
\usepackage{xcolor}

\setcounter{exercise-sheet}{17}

\begin{document}

\begin{center}
{\Large\bf Exercises week \arabic{exercise-sheet}}\\[1ex]
% {\small 2009-08-28}\\[2ex]
\end{center}


\begin{exercise} 
Use the compiler \texttt{scomp} from \texttt{Intcomp1.hs} to compile the following 
expressions to produce a sequence of instructions. You shall try to work it out on paper before checking your answer with GHCi. 
Note that you will need to translate the programs into abstract syntax before calling \texttt{scomp}. 

{\codesetup\begin{verbatim}
   let z = 17 in z + z
\end{verbatim}}

{\codesetup\begin{verbatim}
   20 + let z = 17 in z + 2 end + 30
\end{verbatim}}

Now execute your instruction sequences according to \texttt{seval} by specifying the stack content at each step. Again, you shall work 
it out on paper before using GHCi. Note that if your execution results in an error, 
it may mean that your instruction sequence from above is wrong. 

You are strongly encouraged to repeat this exercise with other expressions you may come out with and check the answer with GHCi.\\

  \noindent
\textbf{\emph{Answer}}
  {\codesetup\begin{verbatim}
  [SCstI 17,SVar 0,SVar 1,SAdd,SSwap,SPop]
  [SCstI 20,SCstI 17,SVar 0,SCstI 2,SAdd,SSwap,SPop,SAdd,SCstI 30,SAdd]
  \end{verbatim}}

\end{exercise}








\begin{exercise}\label{exer-multi-let-tcomp}
In the last lab, we have extend the expression language \texttt{Expr} from \texttt{Intcomp1.hs}
  with multiple \emph{sequential} let-bindings, such as this (in
  concrete syntax):

{\codesetup\begin{verbatim}
   let x1 = 5+7   x2 = x1*2 in x1+x2 end
\end{verbatim}}


  Revise the \texttt{Expr}-to-\texttt{TExpr} compiler \texttt{tcomp ::\ 
    Expr -> TExpr} from \texttt{Intcomp1.fs} to work for the extended
  \texttt{Expr} language.
  There is no need to modify the \texttt{TExpr} language or the
  \texttt{teval} interpreter to accommodate multiple sequential
  let-bindings.\\
  
  \noindent
\textbf{\emph{Answer}}
{\codesetup\begin{verbatim}
  tcomp :: Expr -> [String] -> TExpr 
  tcomp (CstI i) cenv = TCstI i
  tcomp (Var x) cenv = TVar (getindex cenv x)
  tcomp (Let xs ebody) cenv
      = case xs of []              -> tcomp ebody cenv 
                   ((x, erhs):xs') -> let cenv1 = (x:cenv)
                                      in  TLet (tcomp erhs cenv) (tcomp (Let xs' ebody) cenv1)
  tcomp (Prim op e1 e2) cenv
      = TPrim op (tcomp e1 cenv) (tcomp e2 cenv)
\end{verbatim}}
\end{exercise}


\begin{exercise}\label{exer-comp-expr-to-ints}
  Write a bytecode assembler (in Haskell) that translates a list
  of bytecode instructions for the simple stack machine in
  \texttt{Intcomp1.fs} into a list of integers.  The integers should be
  the corresponding bytecodes as seen below.
  {\codesetup\begin{verbatim}
  SCST = 0, SVAR = 1, SADD = 2, SSUB = 3, SMUL = 4, SPOP = 5, SSWAP = 6;\end{verbatim}}
  \noindent 
  Thus you should write a function
  \texttt{assemble ::\ [SInstr] -> [Int]}.\\
  
  \noindent
\textbf{\emph{Answer}}
  {\codesetup\begin{verbatim}
  assemble :: [SInstr] -> [Int]
  assemble [] = []
  assemble (x:xs) 
    = let xs' = assemble xs 
      in case x of SCstI i -> (0:i:xs')
                   SVar  i -> (1:i:xs')
                   SAdd    -> (2:xs')
                   SSub    -> (3:xs')
                   SMul    -> (4:xs')
                   SPop    -> (5:xs')
                   SSwap   -> (6:xs')
  \end{verbatim}}

  Use this function together with \texttt{scomp} from
  \texttt{Intcomp1.hs} to make a compiler from the original expressions
  language \texttt{Expr} to a list of bytecodes \texttt{[Int]}.\\
  
  \noindent
\textbf{\emph{Answer}}
  {\codesetup\begin{verbatim}
  scompeval :: Expr -> [StackValue] -> [Int]
  scompeval x = assemble . scomp x
  \end{verbatim}}

\end{exercise}


\begin{exercise}\label{exer-read-ints-into-stack-machine}
  Modify the compiler from Exercise~\ref{exer-comp-expr-to-ints} to
  write the lists of integers to a file.  A list \texttt{inss} of
  integers may be output to the file called \texttt{fname} using this
  function (found in \texttt{Intcomp1.hs}):

{\codesetup\begin{verbatim}
   intsToFile :: [Int] -> String -> IO ()
   intsToFile inss fname 
    = do let text = intercalate " " (map show inss)
         writeFile fname text
\end{verbatim}}
\noindent
\textbf{\emph{Answer}}
  {\codesetup\begin{verbatim}
  scompeval :: Expr -> [StackValue] -> IO ()
  scompeval x = flip intsToFile "fname" . assemble . scomp x
  \end{verbatim}}

\end{exercise}




\end{document}


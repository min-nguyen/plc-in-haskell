
%       dvips -ta4 exercise01 -o
%       ps2pdf -sPAPERSIZE=a4 exercise01.ps

\documentclass[a4paper]{article}
\usepackage{times,fullpage}
\usepackage[latin1]{inputenc}
\usepackage[T1]{fontenc}
\usepackage{bprd}
\usepackage[dvips]{graphicx}
\usepackage{xcolor}

\setcounter{exercise-sheet}{14}

\begin{document}

\begin{center}
{\Large\bf Exercises week \arabic{exercise-sheet}}\vspace{1cm}
%{\small Last update 2009-08-21}

\end{center}


\begin{exercise}\label{exer-fsharp-functions}
(The purpose of this question is to review Haskell programming. If you are confident with Haskell, skip to the next question.)

  Define the following functions in Haskell:

\begin{itemize}
\item A function \texttt{max2 ::\ Int -> Int -> Int} that returns the
  largest of its two integer arguments.  For instance, \texttt{max2 99
    3} should give 99.\\
  
\noindent
\textbf{\emph{Answer}}    
{\codesetup\begin{verbatim}
max2 :: Int -> Int -> Int
max2 x y = if x > y then x else y
\end{verbatim}}\\
  
\item A function \texttt{max3 ::\ Int -> Int -> Int -> Int} that returns
  the largest of its three integer arguments.\\
    
\noindent
\textbf{\emph{Answer}}
{\codesetup\begin{verbatim}
max3 :: Int -> Int -> Int
max3 x y z = max2 (max2 x y) z
\end{verbatim}}\\
  
\item A function \texttt{isPositive ::\ [Int] -> Bool} so that
  \texttt{isPositive xs} returns true if all elements of \texttt{xs}
  are greater than 0, and false otherwise.\\
    
\noindent
\textbf{\emph{Answer}}
{\codesetup\begin{verbatim}
isPositive :: [Int] -> Bool
isPositive xs = foldr (\x b -> x > 0 && b) True xs
\end{verbatim}}\\
  
\item A function \texttt{isSorted ::\ [Int] -> Bool} so that
  \texttt{isSorted xs} returns true if the elements of \texttt{xs}
  appear sorted in non-decreasing order, and false otherwise.  For
  instance, the list \texttt{[11, 12, 12]} is sorted, but \texttt{[12,
    11, 12]} is not.  Note that the empty list \texttt{[]} and any
  one-element list such as \texttt{[23]} are sorted.\\
    
\noindent
\textbf{\emph{Answer}}
{\codesetup\begin{verbatim}
isSorted :: [Int] -> Bool
isSorted [] = True
isSorted [x] = True
isSorted (x:y:ys) = x <= y && isSorted (y:ys)
\end{verbatim}}\\
  
\item A function \texttt{count ::\ IntTree -> Int} that counts the
  number of internal nodes (\texttt{Br} constructors) in an
  \texttt{IntTree}, where the type \texttt{IntTree} is defined in the
  lecture notes, Appendix A\@.  That is, \texttt{count (Br 37 (Br 117
    Lf Lf) (Br 42 Lf Lf))} should give 3, and \texttt{count Lf}
  should give 0.\\
    
\noindent
\textbf{\emph{Answer}}
{\codesetup\begin{verbatim}
count :: IntTree -> Int
count (Br x l r) = 1 + count l + count r
count Lf = 0
\end{verbatim}}  \\
  
\item A function \texttt{depth ::\ IntTree -> Int} that measures the
  depth of an \texttt{IntTree}, that is, the maximal number of
  internal nodes (\texttt{Br} constructors) on a path from the root to
  a leaf.  For instance, \texttt{depth (Br 37 (Br 117
    Lf Lf) (Br 42 Lf Lf))} should give 2, and \texttt{depth Lf} should give
  0.\\
    
\noindent
\textbf{\emph{Answer}}
{\codesetup\begin{verbatim}
depth :: IntTree -> Int
depth (Br x l r) = 1 + max2 (depth l) (depth r)
count Lf = 0
\end{verbatim}}\\
\end{itemize}
\end{exercise}


\begin{exercise}\label{exer-expr-more-if}
(i) File \texttt{Intro2.hs} on the course homepage contains a definition
of the lecture's \texttt{expr} expression language and an evaluation
function \texttt{eval}.  Extend the \texttt{eval} function to handle
three additional operators: \texttt{"max"}, \texttt{"min"}, and
\texttt{"=="}.  Like the existing operators, they take two argument
expressions.  The
equals operator should return 1 when true and 0 when false.\\
  
\noindent
\textbf{\emph{Answer}}
{\codesetup\begin{verbatim}
eval :: Expr -> [(String, Int)] -> Int 
eval (CstI i) env   = i
eval (Var x)  env   = lookup env x
eval (Prim op e1 e2) env 
    | op == "+" = eval e1 env + eval e2 env 
    | op == "*" = eval e1 env * eval e2 env
    | op == "-" = eval e1 env - eval e2 env
    | op == "max" = let e1' = eval e1 env  
                        e2' = eval e2 env
                    in if e1' > e2' then e1' else e2'
    | op == "min" = let e1' = eval e1 env  
                        e2' = eval e2 env
                    in if e1' < e2' then e1' else e2'
    | op == "=="  = let e1' = eval e1 env  
                        e2' = eval e2 env
                    in if e1' == e2' then 1 else 0
    | otherwise = error "unknown primitive"
\end{verbatim}}\\

\noindent 
(ii) Write some example expressions in this extended expression
language, using abstract syntax, and evaluate them using your new
\texttt{eval} function.\\

\noindent
(iii) Rewrite one of the \texttt{eval} functions to evaluate the
arguments of a primitive before branching out on the operator.\\
  
\noindent
\textbf{\emph{Answer}}
{\codesetup\begin{verbatim}
eval :: Expr -> [(String, Int)] -> Int 
eval (CstI i) env   = i
eval (Var x)  env   = lookup env x
eval (Prim op e1 e2) env 
    = let i1 = eval e1 env
          i2 = eval e2 env
      in case op of
             "+" -> i1 + i2 
             "*" -> i1 * i2
             "-" -> i1 - i2
             _   -> error "unknown primitive"
\end{verbatim}}\\

\noindent
(iv) Extend the expression language with conditional expressions
\texttt{If e1 e2 e3} corresponding to Haskell's conditional expression \texttt{if e1
  then e2 else e3}.
  
You need to extend the \texttt{Expr} datatype with a new constructor
\texttt{If} that takes three \texttt{Expr} arguments.\\
  
\noindent
\textbf{\emph{Answer}}
{\codesetup\begin{verbatim}
data Expr = CstI Int 
          | Var String
          | Prim String Expr Expr  
          | If Expr Expr Expr
          deriving Show
\end{verbatim}}\\

\noindent
(v) Extend the interpreter function \texttt{eval} correspondingly.  It
should evaluate \texttt{e1}, and if \texttt{e1} is non-zero, then
evaluate \texttt{e2}, else evaluate \texttt{e3}.  You should be able
to evaluate this expression \texttt{If (Var "a") (CstI 11) (CstI
  22)} in an environment that binds variable \texttt{a}.\\
      
\noindent
\textbf{\emph{Answer}}
{\codesetup\begin{verbatim}
eval :: Expr -> [(String, Int)] -> Int 
eval (CstI i) env   = i
eval (Var x)  env   = lookup env x
eval (Prim op e1 e2) env 
    = let i1 = eval e1 env
          i2 = eval e2 env
      in case op of
             "+" -> i1 + i2 
             "*" -> i1 * i2
             "-" -> i1 - i2
             _   -> error "unknown primitive"
eval (If e1 then e2 else e3) env
    = if eval e1 != 0
      then e2
      else e3
\end{verbatim}}\\

\end{exercise}



\begin{exercise}\label{exer-aexpr-datatype}
  (i) Declare an alternative datatype \texttt{AExpr} for a
representation of arithmetic expressions without let-bindings.  The
datatype should have constructors \texttt{CstI}, \texttt{Var},
\texttt{Add}, \texttt{Mul}, \texttt{Sub}, for constants, variables,
addition, multiplication, and subtraction.
  
The idea is that we can represent $x*(y+3)$ as \texttt{Mul (Var "x")
  (Add (Var "y") (CstI 3))} instead of \texttt{Prim "*" (Var "x")
  (Prim "+" (Var "y") (CstI 3))}.\\
  
\noindent
\textbf{\emph{Answer}}
{\codesetup\begin{verbatim}
data AExpr = CstI Int
           | Var String
           | Add AExpr AExpr 
           | Mul AExpr AExpr
           | Sub AExpr AExpr
           deriving Show
\end{verbatim}}\\

\noindent
(ii) Write the representation of the expressions $v-(w+z)$ and
$2*(v-(w+z))$ and $x+y+z+v$.\\
  
\noindent
\textbf{\emph{Answer}}
{\codesetup\begin{verbatim}
Sub (Var "v") (Add (Var "w") (Var "z"))
Mul (Const 2) (Sub (Var "v") (Add (Var "w") (Var "z")))
Add (Add (Add (Var "x") (Var "v")) (Var "w")) (Var "z")
\end{verbatim}}\\

\noindent
(iii) Write a Haskell function \texttt{fmt ::\ AExpr -> String } to format
expressions as strings.  For instance, it may format \texttt{Sub (Var
  "x") (CstI 34)} as the string \texttt{"(x - 34)"}.  It has very much
the same structure as an \texttt{eval} function, but takes no
environment argument (because the \emph{name} of a variable is
independent of its \emph{value}).\\
  
\noindent
\textbf{\emph{Answer}}
{\codesetup\begin{verbatim}
fmt :: AExpr -> String
fmt (CstI i) = show i
fmt (Var x)  = x
fmt (Add aexp1 aexp2) = "(" ++ fmt aexp1 ++ " + " ++ fmt aexp2 ++ ")"
fmt (Mul aexp1 aexp2) = "(" ++ fmt aexp1 ++ " * " ++ fmt aexp2 ++ ")"
fmt (Sub aexp1 aexp2) = "(" ++ fmt aexp1 ++ " - " ++ fmt aexp2 ++ ")"
\end{verbatim}}\\

\noindent
(iv) Write a Haskell function \texttt{simplify ::\ AExpr -> AExpr} to
perform expression simplification.  For instance, it should simplify
$(x+0)$ to $x$, and simplify $(1+0)$ to $1$.  The more ambitious
student may want to simplify $(1+0)*(x+0)$ to $x$.  Hint 1: Pattern
matching is your friend.  Hint 2: Don't forget the case where you
cannot simplify anything.

You might consider the following simplifications, plus any others you
find useful and correct:

\begin{displaymath}
  \begin{array}{lcl}\hline\hline
    0 + e & \longrightarrow & e\\
    e + 0 & \longrightarrow & e\\
    e - 0 & \longrightarrow & e\\
    1 * e & \longrightarrow & e\\
    e * 1 & \longrightarrow & e\\
    0 * e & \longrightarrow & 0\\
    e * 0 & \longrightarrow & 0\\
    e - e & \longrightarrow & 0\\\hline\hline
  \end{array}
\end{displaymath}
  
\noindent
\textbf{\emph{Answer}}
{\codesetup\begin{verbatim}
simplify :: AExpr -> AExpr
simplify (CstI i) = CstI i
simplify (Var x)  = Var x
simplify (Add aexp1 aexp2) 
    = let aexp1' = simplify aexp1 
          aexp2' = simplify aexp2 
      in  case (aexp1', aexp2') of (CstI 0, _) -> aexp2' 
                                   (_, CstI 0) -> aexp1'
                                   _           -> Add aexp1' aexp2'
simplify (Sub aexp1 aexp2) 
    = let aexp1' = simplify aexp1 
          aexp2' = simplify aexp2 
      in  if aexp1' == aexp2' 
          then CstI 0 
          else case aexp2' of (CstI 0) -> aexp1' 
                              _        -> Sub aexp1' aexp2'
simplify (Mul aexp1 aexp2) 
    = let aexp1' = simplify aexp1 
          aexp2' = simplify aexp2 
      in  case (aexp1', aexp2') of (CstI 0, _) -> aexp1' 
                                   (_, CstI 0) -> aexp2'
                                   (CstI 1, _) -> aexp2'
                                   (_, CstI 1) -> aexp1'
                                   _           -> Mul aexp1' aexp2'
\end{verbatim}}\\


%\noindent
%(v) [Only for people with fond recollections of differential
%calculus].  Write a Haskell function to perform symbolic differentiation
%of simple arithmetic expressions (such as \texttt{AExpr}) with respect
%to a single variable. \\  
%
%\noindent
%\textbf{\emph{Answer }}{\color{red}{Unsure about answer.}}\\
\end{exercise}



\begin{exercise}\label{exer-better-fmt}
  Write a version of the formatting function \texttt{fmt} from the
preceding exercise that avoids producing excess parentheses.  For
instance,

{\codesetup\begin{verbatim}
   Mul (Sub (Var "a") (Var "b")) (Var "c")
\end{verbatim}}

\noindent    
should be formatted as \texttt{"(a-b)*c"} instead of
\texttt{"((a-b)*c)"}, whereas

{\codesetup\begin{verbatim}
   Sub (Mul (Var "a") (Var "b")) (Var "c")
\end{verbatim}}
    
\noindent    
should be formatted as \texttt{"a*b-c"} instead of
\texttt{"((a*b)-c)"}.  Also, it should be taken into account that
operators associate to the left, so that

{\codesetup\begin{verbatim}
   Sub (Sub (Var "a") (Var "b")) (Var "c") 
\end{verbatim}}

\noindent    
is formatted as \texttt{"a-b-c"} whereas

{\codesetup\begin{verbatim}
   Sub (Var "a") (Sub (Var "b") (Var "c"))
\end{verbatim}}

\noindent    
is formatted as \texttt{"a-(b-c)"}.\\
     
\noindent
\textbf{\emph{Answer}}
{\codesetup\begin{verbatim}
fmt :: AExpr -> String
fmt e = fmt2 (-1) e

fmt2 :: AExpr -> Int -> String
fmt2 (CstI i) ctx_prec = show i
fmt2 (Var x) ctx_prec  = x
fmt2 (Add aexp1 aexp2) ctx_prec = parens ctx_prec 6 (fmt2 5 e1 ++ " + " ++ fmt2 6 e2)
fmt2 (Sub aexp1 aexp2) ctx_prec = parens ctx_prec 6 (fmt2 5 e1 ++ " - " ++ fmt2 6 e2) 
fmt2 (Mul aexp1 aexp2) ctx_prec = parens ctx_prec 7 (fmt2 6 e1 ++ " * " ++ fmt2 7 e2)

parens :: Int -> Int -> String -> String
parens ctx_prec prec aexpr = if ctx_prec < prec 
                             then aexpr 
                             else "(" ++ aexpr ++ ")"
\end{verbatim}}

\end{exercise}

\end{document}


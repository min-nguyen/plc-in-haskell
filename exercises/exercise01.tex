
%       dvips -ta4 exercise01 -o
%       ps2pdf -sPAPERSIZE=a4 exercise01.ps

\documentclass[a4paper]{article}
\usepackage{times,fullpage}
\usepackage[latin1]{inputenc}
\usepackage[T1]{fontenc}
\usepackage{bprd}
\usepackage[dvips]{graphicx}
\usepackage{xcolor}
 
\setcounter{exercise-sheet}{14}

\begin{document}

\begin{center}
{\Large\bf Exercises week \arabic{exercise-sheet}\\[1ex]}\vspace{1cm}
%Monday 24 August 2009}\\[1ex]
%{\small Last update 2009-08-21}\\[2ex]
\end{center}

%\subsection*{Goal of the exercises}
%
%The goal of this week's exercises is to make sure that you have a good
%understanding of functional programming with algebraic datatypes,
%pattern matching and recursive functions.  This is a necessary basis
%for the rest of the course.  Also, you should know how to use these
%concepts for representing and processing expressions in the form of
%abstract syntax trees.
%
%% The following abbreviations are used in exercise sheets:
%
%% \begin{itemize}
%% \item ``Mogensen'' means Torben Mogensen: \emph{Basics of Compiler
%%     Design, Extended edition}, DIKU, University of Copenhagen and
%%   \texttt{lulu.com}, May 2009.
%  
%% \item ``Syme'' means Don Syme, Adam Granicz and Antonio Cisternino:
%%   \emph{Expert F\#}, Apress 2007.
%  
%% \item ``PLCSD'' means Peter Sestoft: \emph{Programming Language
%%     Concepts for Software Developers}, draft manuscript, IT University
%%   2009.
%% \end{itemize}
%
%% \noindent
%
%The exercises let you try yourself the ideas and concepts that were
%introduced in the lectures.  Some exercises may be challenging, but
%they are not supposed to require days of work.  If you get stuck,
%please use the course newsgroup \texttt{it-c.courses.BPRD} to ask for
%help.
%
%It is recommended that you solve
%Exercises~\ref{exer-fsharp-functions}, \ref{exer-expr-more-if},
%\ref{exer-aexpr-datatype} and \ref{exer-datatype-java}, and hand in
%the solutions.  If you solve more exercises, you are welcome to hand
%in those solutions also.
%
%
%\subsection*{Do this first}
%
%At the time of writing, the most recent version of F\# is the May 2009
%Community Technology Preview (CTP), available from
%http://msdn.microsoft.com/fsharp/.  If you are using VS2008, you are
%advised to download and install this version of F\#, because the rest
%of the course (especially the lexing and parsing part) will rely on
%it.
%
%Note that Appendix A of the course lecture notes (see course website)
%contains information on F\# that may be valuable when doing these
%exercises.



\begin{exercise}\label{exer-fsharp-functions}
(The purpose of this question is to review Haskell programming. If you are confident with Haskell, skip to the next question.)

  Define the following functions in Haskell: 

\begin{itemize}
\item A function \texttt{max2 ::\ Int -> Int -> Int} that returns the
  largest of its two integer arguments.  For instance, \texttt{max2 99
    3} should give 99.
  
\item A function \texttt{max3 ::\ Int -> Int -> Int -> Int} that returns
  the largest of its three integer arguments.
  
\item A function \texttt{isPositive ::\ [Int] -> Bool} so that
  \texttt{isPositive xs} returns true if all elements of \texttt{xs}
  are greater than 0, and false otherwise.
  
\item A function \texttt{isSorted ::\ [Int] -> Bool} so that
  \texttt{isSorted xs} returns true if the elements of \texttt{xs}
  appear sorted in non-decreasing order, and false otherwise.  For
  instance, the list \texttt{[11, 12, 12]} is sorted, but \texttt{[12,
    11, 12]} is not.  Note that the empty list \texttt{[]} and any
  one-element list such as \texttt{[23]} are sorted.
  
\item A function \texttt{count ::\ IntTree -> Int} that counts the
  number of internal nodes (\texttt{Br} constructors) in an
  \texttt{IntTree}, where the type \texttt{IntTree} is defined by:
  {\codesetup\begin{verbatim}
    data IntTree = Br Int IntTree IntTree | Lf \end{verbatim}}
  That is, \texttt{count (Br 37 (Br 117
    Lf Lf) (Br 42 Lf Lf))} should give 3, and \texttt{count Lf}
  should give 0.
  
\item A function \texttt{depth ::\ IntTree -> Int} that measures the
  depth of an \texttt{IntTree}, that is, the maximal number of
  internal nodes (\texttt{Br} constructors) on a path from the root to
  a leaf.  For instance, \texttt{depth (Br 37 (Br 117
    Lf Lf) (Br 42 Lf Lf))} should give 2, and \texttt{depth Lf} should give
  0.
\end{itemize}
\end{exercise}


\begin{exercise}\label{exer-expr-more-if}
(i) File \texttt{Intro2.hs} on the course homepage contains a definition
of the lecture's \texttt{expr} expression language and an evaluation
function \texttt{eval}.  Extend the \texttt{eval} function to handle
three additional operators: \texttt{"max"}, \texttt{"min"}, and
\texttt{"=="}.  Like the existing operators, they take two argument
expressions.  The
equals operator should return 1 when true and 0 when false.\\

\noindent 
(ii) Write some example expressions in this extended expression
language, using abstract syntax, and evaluate them using your new
\texttt{eval} function.\\

\noindent
(iii) Rewrite one of the \texttt{eval} functions to evaluate the
arguments of a primitive before branching out on the operator, in this
style:

{\codesetup\begin{verbatim}
eval :: Expr -> [(String, Int)] -> Int 
eval (CstI i) env   = ...
eval (Var x)  env   = ...
eval (Prim op e1 e2) env 
    = let i1 = ...
          i2 = ...
      in case op of
             "+" -> i1 + i2 
             ...
\end{verbatim}}

\noindent
(iv) Extend the expression language with conditional expressions
\texttt{If e1 e2 e3} corresponding to Haskell's conditional expression \texttt{if e1
  then e2 else e3}.
  
You need to extend the \texttt{Expr} datatype with a new constructor
\texttt{If} that takes three \texttt{Expr} arguments.\\

\noindent
(v) Extend the interpreter function \texttt{eval} correspondingly.  It
should evaluate \texttt{e1}, and if \texttt{e1} is non-zero, then
evaluate \texttt{e2}, else evaluate \texttt{e3}.  You should be able
to evaluate this expression \texttt{If (Var "a") (CstI 11) (CstI
  22)} in an environment that binds variable \texttt{a}.
    
Note that various strange and non-standard interpretations of the
conditional expression are possible.  For instance, the interpreter
might start by testing whether expressions \texttt{e2} and \texttt{e3}
are syntactically identical, in which case there is no need to
evaluate \texttt{e1}, only \texttt{e2} (or \texttt{e3}).  Although
possible, this is rarely useful.
\end{exercise}



\begin{exercise}\label{exer-aexpr-datatype}
  (i) Declare an alternative datatype \texttt{AExpr} for a
representation of arithmetic expressions without let-bindings.  The
datatype should have constructors \texttt{CstI}, \texttt{Var},
\texttt{Add}, \texttt{Mul}, \texttt{Sub}, for constants, variables,
addition, multiplication, and subtraction.
  
The idea is that we can represent $x*(y+3)$ as \texttt{Mul (Var "x")
  (Add (Var "y") (CstI 3))} instead of \texttt{Prim "*" (Var "x")
  (Prim "+" (Var "y") (CstI 3))}.\\

\noindent
(ii) Write the representation of the expressions $v-(w+z)$ and
$2*(v-(w+z))$ and $x+y+z+v$.\\

\noindent
(iii) Write a Haskell function \texttt{fmt ::\ AExpr -> String } to format
expressions as strings.  For instance, it may format \texttt{Sub (Var
  "x") (CstI 34)} as the string \texttt{"(x - 34)"}.  It has very much
the same structure as an \texttt{eval} function, but takes no
environment argument (because the \emph{name} of a variable is
independent of its \emph{value}).\\

\noindent
(iv) Write a Haskell function \texttt{simplify ::\ AExpr -> AExpr} to
perform expression simplification.  For instance, it should simplify
$(x+0)$ to $x$, and simplify $(1+0)$ to $1$.  The more ambitious
student may want to simplify $(1+0)*(x+0)$ to $x$.  Hint 1: Pattern
matching is your friend.  Hint 2: Don't forget the case where you
cannot simplify anything.

You might consider the following simplifications, plus any others you
find useful and correct:

\begin{displaymath}
  \begin{array}{lcl}\hline\hline
    0 + e & \longrightarrow & e\\
    e + 0 & \longrightarrow & e\\
    e - 0 & \longrightarrow & e\\
    1 * e & \longrightarrow & e\\
    e * 1 & \longrightarrow & e\\
    0 * e & \longrightarrow & 0\\
    e * 0 & \longrightarrow & 0\\
    e - e & \longrightarrow & 0\\\hline\hline
  \end{array}
\end{displaymath}

%\noindent
%(v) [Only for people with fond recollections of differential
%calculus].  Write a Haskell function to perform symbolic differentiation
%of simple arithmetic expressions (such as \texttt{AExpr}) with respect
%to a single variable.\\
\end{exercise}



\begin{exercise}\label{exer-better-fmt}
  Write a version of the formatting function \texttt{fmt} from the
preceding exercise that avoids producing excess parentheses.  \\

\noindent For instance,

{\codesetup\begin{verbatim}
   Mul (Sub (Var "a") (Var "b")) (Var "c")
\end{verbatim}}

\noindent    
should be formatted as \texttt{"(a-b)*c"} instead of
\texttt{"((a-b)*c)"}, whereas

{\codesetup\begin{verbatim}
   Sub (Mul (Var "a") (Var "b")) (Var "c")
\end{verbatim}}
    
\noindent    
should be formatted as \texttt{"a*b-c"} instead of
\texttt{"((a*b)-c)"}. \\

\noindent Also, it should be taken into account that
operators associate to the left, so that

{\codesetup\begin{verbatim}
   Sub (Sub (Var "a") (Var "b")) (Var "c") 
\end{verbatim}}

\noindent    
is formatted as \texttt{"a-b-c"} whereas

{\codesetup\begin{verbatim}
   Sub (Var "a") (Sub (Var "b") (Var "c"))
\end{verbatim}}

\noindent    
is formatted as \texttt{"a-(b-c)"}.\\
   
\noindent Hint: This can be achieved by declaring the formatting function to
take an extra parameter \texttt{pre} that indicates the precedence or
binding strength of the context.  The new formatting function then has
type \texttt{fmt ::\ AExpr -> Int -> String}.
  
Higher precedence means stronger binding.  When the top-most operator
of an expression to be formatted has higher precedence than the
context, there is no need for parentheses around the expression.  A
left associative operator of precedence 6, such as minus (\texttt{-}),
provides context precedence 5 to its left argument, and context
precedence 6 to its right argument.

As a consequence, \texttt{Sub (Var "a") (Sub (Var "b") (Var "c"))} will be
parenthesized \texttt{a - (b - c)} but \texttt{Sub (Sub (Var "a") (Var "b")) (Var "c")} will be parenthesized \texttt{a - b - c}.
\end{exercise}

\end{document}

\begin{exercise}\label{exer-datatype-java}
{\color{red}{This exercise uses Java/C\# - I haven't translated this as I'm unsure whether the exercise is necessary.}}  
Lecture 1 shows how to represent abstract syntax in functional
languages such as F\# (using algebraic datatypes) and in
object-oriented languages such as Java or C\# 
(using a class hierarchy and composites).\\

\noindent
(1) Use Java (or C\#) classes and methods to do what we have done
using the F\# datatype \texttt{aexpr} in the preceding exercises.
Design a class hierarchy to represent arithmetic expressions: it could
have an abstract class \texttt{Expr} with subclasses \texttt{CstI},
\texttt{Var}, and \texttt{Binop}, where the latter is itself abstract
and has concrete subclasses \texttt{Add}, \texttt{Mul} and
\texttt{Sub}\@.  All classes should implement the \texttt{toString()}
method to format an expression as a \texttt{String}.

The classes may be used to build an expression in abstract syntax, and
then print it, as follows:

{\codesetup\begin{verbatim}
   Expr e = new Add(new CstI(17), new Var("z"));
   System.out.println(e.toString());
\end{verbatim}}

\noindent
(ii) Create three more expressions in abstract syntax and print them.\\

\noindent
(iii) Extend your classes with facilities to evaluate the arithmetic
expressions, that is, add a method \texttt{int eval(env)}.

\end{exercise}




The class
declarations may look like this:

{\codesetup\begin{verbatim}
   abstract class Expr { 
     abstract public String toString();
   }
   
   class CstI extends Expr { 
     protected final int i;
   
     public CstI(int i) { 
       this.i = i; 
     }
   
     public String toString() { ... } 
   }
   
   class Var extends Expr { 
     protected final String name;
   
     public Var(String name) { 
       this.name = name; 
     }
   
     public String toString() { ... } 
   }
   
   abstract class Binop extends Expr { 
     protected final String oper;
     protected final Expr e1, e2;
   
     public Binop(String oper, Expr e1, Expr e2) { 
       this.oper = oper; this.e1 = e1; this.e2 = e2;
     }
   
     public String toString() { ... } 
   }
   
   class Add extends Binop { ... }   
   class Mul extends Binop { ... }    
   class Sub extends Binop { ... }    
   class Div extends Binop { ... }    
\end{verbatim}}
